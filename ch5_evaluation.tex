\cleardoublepage
\chapter{Implementation and Evaluation}
\label{eval_chapter}

\section{Implementation}

The \todo{protocol name} implementation has a client-side library~\citep{pvc-client} and a server~\citep{pvc-server}, written as a plug-in transactional protocol for AntidoteDB~\citep{antidote-db}, a key-value database. Both implementations were written in the Erlang programming language, with a total of 6K lines of code. AntidoteDB provides the application programmer rich data types such as \todo{list, mention CRDTs}, supports both in-memory and disk-based storage, and implements \todo{full/partial} replication. In our implementation, we constrain ourselves to a simple in-memory, key-value storage using Last-Write-Wins registers (LWWs) \todo{cite?}. In addition, we have disabled replication, and only focus on the distribution problem. During normal operation, the client-side library communicates with the server using Google's Protocol Buffers~\citep{protobuf} through an optimised TCP connection that allows pipelining \todo{Maybe omit}.

We also implemented two alternative protocols that implement different consistency criteria: Read Committed (RC) and Serialisability (SER) \todo{cite}, so we can compare our experimental results.

To prevent unbounded state growth, we also implement a simple garbage collection mechanism for both $\CommitLog$ and $\VersionLog$ at each partition. \todo{More? Mention RC and SER implementations. Maybe pseudocode of the implementations can go in an appendix}.

\section{Evaluation}

\todo{Maybe we don't need these many subsections, and can summarise some of them in separate paragraphs}

\subsection{Experimental Setup}

\todo{Details of computers: Xeon E2186G, 6 cores, 12 threads. 3.80-4.70GHz, 32GB memory, 1GbE ethernet. For each server machine we use a client machine. Latency was added using tc. Preloaded with 1 million keys, payload of 256 bytes}

% Links (also see email)
% https://ark.intel.com/content/www/us/en/ark/products/134855/intel-xeon-e-2186g-processor-12m-cache-up-to-4-70-ghz.html

% https://www.supermicro.com/en/products/system/3U/5039/SYS-5039MC-H12TRF.cfm

\subsection{Throughput Micro-benchmark}

\subsection{Dynamic Workloads}

\subsection{Read Abort Rate Impact}

\todo{Maybe?}
