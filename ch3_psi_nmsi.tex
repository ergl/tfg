\cleardoublepage
\chapter{PSI/NMSI}

\todo{
We explain what PSI is, how it differs from classical SI, and how it avoids some of the anomalies explained during the preliminaries. Explain the protocol in terms of abstract executions. As Alexey for reference showing that PSI is equivalent to NMSI.

PSI = SI within a site, (isolation) + causal consistency across sites (consistency) => ref Lamport.

Traditional SI = Sequential consistency?.

Mention~\citep{psi-intro}, difference: transactions are not fixed at start time, but at read time (like NMSI).

From same paper:

Snapshot isolation is inadequate for a system replicated at many sites, due to two issues. First, to define snapshots, snapshot isolation imposes a total ordering of the commit time of all transactions, even those that do not conflict. Establishing such an ordering when transactions execute at different sites is inefficient. Second, the writes of a committed transaction must be immediately visible to later transactions. Therefore a transaction can commit only after its writes have been propagated to all remote replicas, thereby precluding asynchronous propagation of its updates.
}

\todo{
\begin{itemize}
    \item Specification of SI
    \item Specification of PSI (SI requires global sync for total orderings, so we allow different orderings at different sites)
    \item Specification of NMSI (PSI timestamp is fixed at tx.start() time (base freshness), which causes stale transactions in geo-replicated settings.)
\end{itemize}

In chapter 2 we can give an overview, and in this chapter we use Alexey's paper to formalize them. Use an example in chapter 2.

Here, more examples on how traces either satisfy or violate the consistency guarantees. Examples of short fork, long fork, and how big of a deal are those anomalies in our target applications (a good exaple is Sovran's Figure 8).
}