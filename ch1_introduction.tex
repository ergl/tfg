\cleardoublepage
\chapter{Introduction}

\todo{
Systems that offer strong consistency are usually slow, so we focus on a model between strong and weak consistency, and show that an implementation can achieve performance close to the offered by weak consistent systems, but without their limitations.

Mention previous work (NMSI), and how we differ from them (vector clocks vs dependence vectors)

Overview of what we're covering in the thesis: explanation of the model (PSI/NSMI), an implementation of the protocol, and a comparative evaluation. Link to the code on github.

Tip: We can be a bit ``handwavy'' with terminology here (meaning use something without defining it too formally). We can dive deeper into terms in the preliminaries section.
}

The rise of ubiquitous, globally-available services on the Internet, paired with the ever-increasing amount of data being uploaded to these services, presents two problems: on the one hand, no one service can store all its data in a single computer; on the other hand, users expect that these services operate in a responsive and reliable manner. The solutions are \emph{distribution}, which involves dividing large amounts of data in \emph{shards} or \emph{partitions} among multiple computers; and \emph{replication}, which allows to place identical copies of data in multiple geographical regions. Modern cloud applications use a combination of the two: for example, a social media site might decide to partition its data by user location, placing their data in geographical regions closest to these locations, so that users can access their data without incurring a high latency penalty. In addition, the site might replicate these partitions to other geographical regions to achieve fault-tolerance: no single faulty region can result in data being unavailable.

These approaches, however, add significant complexity to the design, implementation and usage of the databases that usually power cloud applications. The presence of multiple replicas raises the question of how to keep them \emph{consistent}, that is, reflecting an unified version of the data they contain. Traditional relational databases also implement \emph{transactions}, that allow application programmers to reason about their code as a set of \emph{isolated}, atomic operations on shared data. With partitioned databases, transactions might be \emph{distributed}, updating values in multiple partitions, and therefore requiring coordination between them so that changes are applied to the database in an atomic manner.

\todo{Mention CAP, given that network partitions are already chosen for us. (see COPS paper CAP refs, maybe add COPS to refs) Traditional approach is to weaken consistency, but it's hard to reason about (mention problems, like lack of transactions). Say what consistency actually is}

\todo{At some point, we should distinguish between isolation and consistency, but that can go in next chapter}

\todo{Maybe define ``consistency guarantees'' first}
Traditional approaches to bridge these problems involve relaxing the consistency guarantees that these databases offer the programmers~\citep{vogels-eventual}. Indeed, the CAP Theorem~\citep{cap-brewer, cap-theorem} proves it is impossible to build applications that continue operating in the presence of network partitions without sacrificing consistency guarantees. However, the degree to which this guarantees can be relaxed offers a trade-off: on the one hand, weak consistency guarantees allow to build scalable applications without loss of availability, but prove difficult to reason about, and force programmers to deal with inconsistent data at the application level; on the other hand, strengthening consistency guarantees can reduce performance and hurt application availability, while offering the programmers better ways to build applications, like transactions.

\todo{Mention causal consistency, good approach to build social media applications. Range from SSER to RC. Both PSI/NMSI offer good trade-offs.}

\todo{Contrast PSI and NSMI, say what we implemented. Say how we differ.}

\todo{Contents of each chapter, link to code}

\todo{(from~\citep{psi-intro}): Storage systems that span many data centers often do not pro- vide transactions (e.g., Dynamo [16]), or support only restricted transactional semantics. For example, PNUTS [14] supports only one-record transactions. COPS [31] provides only read-only trans- actions. Megastore [7] partitions data and provides the ACID prop- erties within a partition but, unlike Walter, it fails to provide full transactional semantics for reads across partitions.

Build on this to explain motivation}
