\newpage

\begin{center}
{\bf \Huge Resumen}
\end{center}

\vspace{1cm}

La mayoría de las bases de datos distribuidas usan protocolos de consistencia débil
para evitar la penalización de rendimiento que supone la coordinación de las distintas
réplicas. Sin embargo, dichos protocolos imponen en los programadores la necesidad de
razonar sobre posibles anomalías, así como implementar mecanismos de resolución de
conflictos en el código de las aplicaciones.

El protocolo PSI (Parallel Snapshot Isolation) se ha propuesto como una posible solución,
sin embargo sufre de una significativa tasa de abortos debido a lecturas de datos antiguos,
dado que las transaciones están obligadas a leer de una instantánea fijada en el momento
en que comienzan.

En contraste, el protocolo NMSI (Non-Monotonic Snapshot Isolation) propuesto recientemente
permite a las transacciones leer versiones de datos confirmados después de su comienzo,
lo que conlleva una tasa de cancelación más baja y una mayor escalabilidad. A pesar de esto,
la implementación propuesta utiliza un mecanismo de relojes complejo para garantizar instantáneas
consistentes.

En este trabajo, mostraremos una implementación del protocolo NMSI usando relojes lógicos
para garantizar instantáneas consistentes y una resolución de conflictos simple, comparándola
con la implementación anterior. Además, realizamos un estudio comparativo de diferentes
implementaciones de protocolos de consistencia, mostrando que NMSI puede ofrecer un rendimiento
similar a los protocolos más débiles mientras proporciona garantías mas fuertes.

\vspace{1cm}

\begin{center}
{\bf \Large Palabras clave}
\end{center}

\vspace{0.5cm}

% TODO(borja): Fill
Modelos de Consistencia, Transacciones, Control de Concurrencia.
