\newpage

\begin{center}
{\bf \Huge Resumen}
\end{center}

\vspace{1cm}

La mayoría de las bases de datos distribuidas ofrecen modelos de consistencia
débil, con la finalidad de evitar la penalización de rendimiento que supone la
coordinación de las distintas réplicas. Idealmente, las bases de datos distribuidas
ofrecerían modelos de consistencia fuerte, como \emph{serialisability}, ya que
facilitan la verificación de los invariantes de las aplicaciones, y permiten que
los programadores no deban preocuparse sobre posibles problemas de concurrencia.
Sin embargo, implementar sistemas escalables que con modelos de consistencia
fuerte no es fácil, pues requieren el uso de comunicación global. Sin embargo,
aunque los modelos de consistencia más débiles permiten sistemas más escalables,
imponen en los programadores la necesidad de razonar sobre posibles anomalías,
así como implementar mecanismos de resolución de conflictos en el código de las
aplicaciones.

Dos modelos de consistencia propuestos recientemente, Parallel Snapshot Isolation
(PSI) y Non-Monotonic Snapshot Isolation (NMSI), representan los modelos más
fuertes que permiten implementaciones escalables sin necesidad de comunicación
global. Permiten, a su vez, implementar sistemas con rendimientos similares a
aquellos con modelos más débiles, a la vez que mantienen tasas de cancelación de
transacciones similares. Aun así, ambos modelos no logran ofrecer las mismas
garantías que \emph{serialisability}, por lo que pueden ser difíciles de usar desde el
punto de vista de las aplicaciones.

Este trabajo presenta una propuesta que busca acortar la distancia entre
modelos como PSI y NMSI y modelos fuertes como \emph{serialisability}. Con esa
finalidad, este trabajo presenta fastPSI, un protocolo de consistencia que permite
al usuario ejecutar de manera selectiva transacciones serializables, reteniendo a
su vez las propiedades de escalabilidad propias de modelos de consistencia débiles
como PSI o NMSI. Además, este trabajo cuenta con una evaluación exhaustiva de
fastPSI, comparándolo con otros protocolos de consistencia, tanto fuertes como
débiles. Se muestra así que fastPSI logra un rendimiento mayor que
\emph{serialisability} sin por ello renunciar a la escalabilidad de protocolos
más débiles.

\vspace{1cm}

\begin{center}
{\bf \Large Palabras clave}
\end{center}

\vspace{0.5cm}

Modelos de Consistencia, Transacciones, Parallel Snapshot Isolation,
Non-Monotonic Snapshot Isolation, Control de Concurrencia.
