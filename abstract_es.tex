\newpage

\begin{center}
{\bf \Huge Resumen}
\end{center}

\vspace{1cm}

La mayoría de las bases de datos distribuidas ofrecen modelos de consistencia débil
para evitar la penalización de rendimiento que supone la coordinación de las distintas
réplicas. Sin embargo, dichos modelos imponen en los programadores la necesidad de
razonar sobre posibles anomalías y de implementar mecanismos de resolución de
conflictos en el código de las aplicaciones.

El modelo clásico de consistencia fuerte, \emph{serialisability}, facilita la
verificación de los invariantes de las aplicaciones, y permite que los programadores
no se preocupen sobre posibles problemas de concurrencia. Sin embargo, las bases de datos
que ofrecen serialisability son difíciles de implementar y escalar, pues normalmente
requieren el uso de comunicación global.

Dos modelos de consistencia propuestos recientemente, Parallel Snapshot Isolation
(PSI) y Non-Monotonic Snapshot Isolation (NMSI), permiten que las bases de datos
distribuidas tengan un mayor rendimiento y una menor tasas de cancelación de
transacciones, siendo a su vez los modelos de consistencia más fuertes que permiten
implementaciones escalables que no requieren comunicación global. Aun así,
ambos modelos no ofrecen las mismas garantías que serialisability, por lo que
pueden ser difíciles de usar desde el punto de vista de las aplicaciones.

En este trabajo, mostramos una propuesta que busca acortar la distancia entre los modelos
de consistencia fuerte y débil. Implementamos fastPSI, un protocolo de consistencia
que permite al usuario ejecutar de manera selectiva transacciones serializables,
reteniendo a su vez las propiedades de escalabilidad propias de modelos de consistencia
débiles como PSI o NMSI. Realizamos una evaluación exhaustiva de fastPSI, comparándolo
con otros protocolos de consistencia, tanto fuertes como débiles. Mostramos así
que nuestra implementación logra un rendimiento mayor que serialisability sin por
ello renunciar a la escalabilidad de protocolos más débiles.

\vspace{1cm}

\begin{center}
{\bf \Large Palabras clave}
\end{center}

\vspace{0.5cm}

Modelos de Consistencia, Transacciones, Parallel Snapshot Isolation, Non-Monotonic Snapshot Isolation, Control de Concurrencia.
