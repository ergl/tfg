\newpage

\begin{center}
{\bf \Huge Abstract}
\end{center}

\vspace{1cm}

Most distributed database systems offer weak consistency models in order to avoid
the performance penalty of coordinating replicas. Ideally, distributed databases
would offer strong consistency models, like serialisability, since they make it
easy to verify application invariants and frees programmers from worrying about
concurrency. However, implementing and scaling systems with strong consistency
is difficult, since it usually requires global communication. Weak models, while
easier to scale, impose on the programmers the need to reason about possible
anomalies, and the need to implement conflict resolution mechanisms in application
code.

Recently proposed consistency models, like Parallel Snapshot Isolation (PSI) and
Non-Monotonic Snapshot Isolation (NMSI), represent the strongest models that still
allow to build scalable systems without global communication. They allow comparable
performance to previous, weaker models, as well as similar abort rates. However, both
models still provide weaker guarantees than serialisability, and may prove difficult
to use in applications.

In this work, we show an approach to bridge the gap between PSI, NMSI and strong
consistency models like serialisability. We implement fastPSI, a consistency protocol
that allows the user to selectively enforce serialisability for certain executions,
while retaining the scalability properties of weaker consistency models like PSI and
NMSI. We perform a comprehensive evaluation of fastPSI in comparison with other
consistency protocols, both weak and strong, showing that our implementation
offers better performance than serialisability, while retaining the scalability
of weaker protocols.

\vspace{1cm}

\begin{center}
{\bf \Large Keywords}
\end{center}

\vspace{0.5cm}

Consistency models, Transactions, Parallel Snapshot Isolation, Non-Monotonic
Snapshot Isolation, Concurrency control.
