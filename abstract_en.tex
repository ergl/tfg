\newpage

\begin{center}
{\bf \Huge Abstract}
\end{center}

\vspace{1cm}

\todo{Reword}

Most distributed database systems use weak consistency protocols to avoid the
performance penalty of coordinating replicas. However, these protocols impose
on the programmers the need to reason about possible anomalies, and the need
to implement conflict resolution mechanisms in application code.

Parallel Snapshot Isolation (PSI) has been proposed as a solution to this
problem, however, it suffers a significant abort rate due to stale reads,
given that transactions are forced to read from a fixed snapshot of versions
once they start.

In contrast, the recently proposed Non-Monotonic Snapshot Isolation (NMSI)
protocol allows transactions to read versions of data committed after it
started, leading to a lower abort rate and increased scalability. In spite
of this, the proposed implementation uses a complex clock mechanism to ensure
consistent snapshots.

In this work, we'll show a implementation of the NMSI protocol using logical
clocks to ensure consistent snapshots and simple conflict resolution, and
compare it against the previous implementation. In addition, we perform a
comparative study of different consistency protocol implementations, showing
that NMSI can offer similar performance to weaker protocols while providing
stronger guarantees.

\vspace{1cm}

\begin{center}
{\bf \Large Keywords}
\end{center}

\vspace{0.5cm}

Consistency models, Transactions, Parallel Snapshot Isolation, Non-Monotonic Snapshot Isolation, Concurrency control.
