\newpage

\begin{center}
{\bf \Huge Abstract}
\end{center}

\vspace{1cm}

Most distributed database systems offer weak consistency models in order to avoid
the performance penalty of coordinating replicas. However, these models impose
on the programmers the need to reason about possible anomalies, and the need
to implement conflict resolution mechanisms in application code.

The classic strong consistency model, serialisability, makes it easy to verify
application invariants, and frees programmers from worrying about concurrency.
Nevertheless, implementing and scaling distributed databases that offer
serialisability proves difficult, as it usually requires global communication.

In contrast, the recently proposed Parallel Snapshot Isolation (PSI) and
Non-Monotonic Snapshot Isolation (NMSI) consistency models allow greater
performance and lower abort rates, while still being the strongest models that
allow to build scalable systems without requiring global communication.
On the other hand, these models still provide weaker guarantees than
serialisability, and may prove difficult to use in applications.

In this work, we show an approach to bridge the gap between weak and strong
consistency models. We implement fastPSI, a consistency protocol that allows
the user to selectively enforce serialisability for certain executions, while
retaining the scalability properties of weaker consistency models like PSI and
NMSI. We perform a comprehensive evaluation of fastPSI in comparison with other
consistency protocols, both weak and strong, showing that our implementation
offers better performance than serialisability, while retaining the scalability
of weaker protocols.

\vspace{1cm}

\begin{center}
{\bf \Large Keywords}
\end{center}

\vspace{0.5cm}

Consistency models, Transactions, Parallel Snapshot Isolation, Non-Monotonic Snapshot Isolation, Concurrency control.
