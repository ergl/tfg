\cleardoublepage
\chapter{Serialisable and Read Committed Protocols}
\label{appendix:code}

In this appendix we give a quick overview of the protocols we implemented to compare against the fastPSI implementation. For simplicity, both protocols are minor modifications of the original one, although the implementations are still efficient. For each protocol, we summarise the most relevant changes with respect to fastPSI, and give the full pseudocode. The implementation for both protocols is freely available of GitHub~\citep{pvc-client, pvc-server}.

\section{Serialisability}
\label{appendix:ser}

Recall from~\ref{sect:ser} that under serialisability transactions appear to execute one after the other. However, it does not guarantee a real-time order among them: an implementation is free to reorder the commit order of transactions as long as the resulting execution still satisfies serialisability. Since our original protocol guarantees Parallel Snapshot Isolation, it is sufficient to prevent the write skew and the long fork anomalies. To do so, we expand fastPSI with several changes. A summary of the state of a server is reflected in Figure~\ref{fig:ser-prot-ds-table}, and the pseudocode of the protocol can be found in Figures~\ref{fig:ser_init}--\ref{fig:ser_termination}.

\paragraph{Track versions of read objects.} Read-only transactions need to observe consistent data that it's not too stale, or overwritten by concurrent transactions. In order to enforce this guarantee, we add a new piece of information to the transaction context: the \emph{read-set} of a transaction keeps track of the objects that it reads, along with the commit time of the version of the object read. When a client receives a $\READRETURN$ message after issuing a read of an object $x$ on behalf of a transaction $\tx$, it incorporates the commit time of $x$ in $\tx$'s read-set, $\tx.\RS$ (line~\ref{alg:ser_rs_update}). It suffices to store the $j$-th entry of $x$'s \emph{commit vector}, that is, the \emph{sequence number} assigned to the transaction that wrote the version of $x$ read by $\tx$. The read-set of a transaction is initialised to the empty set (line~\ref{alg:ser_start_tx}) when the transaction starts.

\paragraph{Avoid stale reads.} Recall Recall from Section~\ref{sect:psi} that the long fork anomaly occurs whenever concurrent transactions are able to observe different orderings of non-conflicting transactions. Such anomaly can be precluded by forcing transactions to observe the latest version of an object\todonote{Add example, can we reuse figure~\ref{fig:long_fork_history}?}. The protocol enforces this property during the validation of a transaction $T$. When a transaction $T$ prepares to commit, it performs a two-phase commit among the servers storing the objects read and written by the transaction (lines~\ref{alg:ser_send_prepare}--\ref{alg:ser_send_prepare_2}). Upon receiving a $\PREPARE$ message, a server $s_i$ checks that, for the objects that a $\tx$ read, the version of those objects is the most up-to-date in the database of $s_i$ (line~\ref{alg:ser_conflict_check}, last condition). It does so by comparing the version of the object stored in the transaction's read-set ($\localRS$) against the latest version as reflected by the $\VersionLog$ of the object ($\VersionLog[x].\last.\Vcomm$). The server also needs to perform the check against concurrently-committing transaction (line~\ref{alg:ser_conflict_check}, first condition): a server checks that the read-set of a transaction $\tx$ does not overlap with the write-set of any other concurrently-committing transactions, thereby making $\tx$'s read stale. If the version that $\tx$ read is stale, the server $s_i$ votes $\abort$ (line~\ref{alg:ser_vote_abort}). Since we assume that transactions read an object before writing to it (and thus, $\tx.\WS \subseteq \tx.\RS$) the stale read check also precludes stale writes.

\paragraph{Avoid read-write conflicts.} We can preclude the write skew anomaly, introduced in Section~\ref{sect:si}, by forcing transactions to observe the writes of concurrent transactions as soon as those updates are reflected in the state of the database. A history that exhibits a write skew, such as $h = r_1(x_0).r_1(y_0).r_2(x_0).r_2(y_0).w_1(x_1).w_2(y_2).c_1.c_2$, can only be serialisable by aborting either $\tx_1$ or $\tx_2$. In the protocol, this is avoided by precluding \emph{read-write} conflicts: in the previous history, $\tx_2$ \emph{overwrites} the version of $y$ that $\tx_1$ read. Also, $\tx_1$ overwrites the version of $x$ that $\tx_2$ reads, which is considered a read-write conflict, too. Both examples of conflict can be precluded during the validation of a transaction (line~\ref{alg:ser_conflict_check}, first condition): a server checks that the write-set of a transaction $\tx$ does not overlap with the read-set of any other concurrently-committing transactions (thereby invalidating the read of such concurrent transaction). If any of the checks is true, the server votes $\abort$ (line~\ref{alg:ser_vote_abort}).

\begin{figure}[htb!]
\noindent\adjustbox{max width=\paperwidth}{\footnotesize
\begin{tabularx}{\linewidth}{|c|p{7cm}|X|}
  \hline
  \multicolumn{3}{|c|}{\textbf{Variables at a server $s_i$}} \\
  \hline
  \textbf{Name} & \textbf{Domain} & \textbf{Description} \\
  \hline
  $\lastprep$ & {\sf Integer} & The number of update transactions that tried to
  commit at the server.
\\
  \hline
  $\VersionLog$ & ${\sf Map}[\keytype,$ ${\sf
    Set}[\langle \valuetype\ \val, \vctype\ \Vcomm\rangle]]$ & Database:
  a mapping from objects to lists of pairs of a value and the
  commit vector of the transaction that wrote it. The lists are ordered
  by the $i$-th component of the commit vectors.
\\
  \hline
  $\CommitLog$
  & ${\sf Sequence}[\langle\transtype\ T,$ $\vctype\ \Vaggr \rangle]$
  & Log of update transactions $T$ committed at the server, ordered by
  $\Vaggr[i]$. Here $\Vaggr$ is the aggregate vector of $T$: the join of the
  commit vectors of all transactions up to $T$ in $\CommitLog$.
\\
  \hline
  $\LocalTime$ & $\vctype$ & The join of the commit vectors of all
  transactions in $\CommitLog$.
\\
  \hline
  $\CommitQueue$

  & Sequence$[\langle \transtype, {\sf State}, {\sf ReadSet}, {\sf WriteSet} \rangle]$ where ${\sf State}=\{\pending,\ready\}$

  & Queue containing information about update transactions trying to commit at the server.
\\
  \hline
  \hline
  \multicolumn{3}{|c|}{\textbf{Context for a transaction $T$ at a client $c_i$}} \\
  \hline
  $T.\RS$ & ${\sf ReadSet}$ & Read-set of $T$.
\\
  \hline
  $T.\WS$ & ${\sf WriteSet}$ & Write-set of $T$.
\\
  \hline
  $T.\hasRead$ & ${\sf Vector}[{\sf Bool}]$ & Mapping showing whether $T$ has
  read a given partition.
\\
  \hline
  $T.\VCaggr$ & $\vctype$ & Snapshot vector: determines snapshots fixed at
  partitions $T$ has read from and possible causal dependencies at all other
  partitions.
\\
  \hline
  $T.\VCdep$ & $\vctype$ & Dependency vector, representing all causal
  dependencies developed by $T$ during its execution.
\\
  \hline
\end{tabularx}
}
\caption{List of variables used in the Serialisable protocol, where ${\sf ReadSet} = {\sf Set}[\langle \keytype, {\sf Integer}\rangle]$ and ${\sf WriteSet} = {\sf Set}[\langle \keytype, \valuetype \rangle]$}
\label{fig:ser-prot-ds-table}
\end{figure}

\begin{figure}[h]
\begin{algorithm}[H]
  \setcounter{AlgoLine}{0}
  % Start
  \SubAlgo{\Fun ${\tt start}()$}{
    \Return{$\KwSty{new}\ \transtype(
      \WS= \emptyset,
      \RS= \emptyset,
      \hasRead = \vec{\bot},
      \VCaggr = \vec{0},
      \VCdep = \vec{0})$
    };\label{alg:ser_start_tx}
  }

  \smallskip

  % Write
  \SubAlgo{\Fun ${\tt write}(T, x, v)$}{
    $\tx.\WS \leftarrow \left(\tx.\WS\ \backslash\ \{\langle x, \_ \rangle\}\right) \cup \{\langle x,v\rangle\}$\;
  }
\end{algorithm}
\caption{Initialisation of a transaction and update of an object \emph{x} at client $c_i$ under Serialisability.}
\label{fig:ser_init}
\end{figure}

\begin{figure}[h]
\begin{algorithm}[H]
  % Read
  \SubAlgo{\Fun ${\tt read}(T, x)$}{
    \If{$\langle x, v \rangle \in \tx.\WS$}{
      \Return{$v$}\;
    }

    $j \leftarrow \partitionof(x)$\;
    \Send{$\READREQUEST(x, T.\VCaggr, T.\hasRead)$} \KwTo $s_j$\;
    \Receive{$\READRETURN(m)$} \KwFrom $s_j$\;
    \uIf{$m = \abort$} {
      \Throw{$\abort$}\;
    }
    \ElseIf{$m = \langle v,\localVdep,\localVaggr \rangle$}{
      $\tx.\hasRead[j] \leftarrow \true$\;
      $\tx.\RS \leftarrow
        \left(\tx.\RS\ \backslash\ \{\langle x, \_ \rangle\}\right)
        \cup
        \{\langle x, \localVdep[j] \rangle\}$\;\label{alg:ser_rs_update}
      $\tx.\VCdep \leftarrow \max(\tx.\VCdep,\localVdep)$\;
      $\tx.\VCaggr \leftarrow \max(\tx.\VCaggr,\localVaggr)$\;
      \Return{$v$}\;
    }
  }

  \smallskip

  % ReadRequest
  \SubAlgo{\WhenReceived $\READREQUEST(x, \argVCaggr, \argHasRead)$ \KwFrom $c_j$}{
    \uIf{$\argHasRead[i]$} {
      $V \leftarrow \argVCaggr$\;
    }
    \Else{
      \Until{$\mrvc[i] \ge \argVCaggr[i]$}\;
      $r \leftarrow \max\{r \in \CommitLog \mid \forall j.\, \argHasRead[j] {\implies} \left(r.\Vaggr[j] \le \argVCaggr[j]\right)\}$\;
      \If{$r.\Vaggr[i] < \argVCaggr[i]$}{
        \Send{$\READRETURN(\abort)$} \KwTo $c_j$\;
        \Return\;
      }
      $V \leftarrow r.\Vaggr$\;
    }
    $\ver = \max\{\ver \in \VersionLog \mid ver.\Vcomm[i] \le V[i]\}$\;
    \Send{$\READRETURN(\ver.\val, \ver.\Vcomm,V)$} \KwTo $c_j$\;
  }
\end{algorithm}
\caption{Serialisable local and remote read of object \emph{x}}
\end{figure}

\begin{figure}[h]
\begin{algorithm}[H]
  % Commit
  \SubAlgo{\Fun ${\tt commit}(T)$\label{alg:ser_commit_start}}{
    \ForAll{$\partj \in \partitions(\tx.\RS \cup \tx.\WS)$\label{alg:ser_send_prepare}}{
      \Send{$\PREPARE(T, T.\RS, T.\WS, \VCdep)$} \KwTo $\partj$\;\label{alg:ser_send_prepare_2}
    }

    $\commitVC \leftarrow \tx.\VCdep$\;
    $\outcome \leftarrow \commit$\;

    \ForAll{$\partj \in \partitions(\tx.\RS \cup \tx.\WS)$}{
      \Receive{$\VOTE(m)$} \KwFrom $\partj$\;
      \uIf{$m = \langle T, \abort \rangle$}{
        $\outcome \leftarrow \abort$\;
        \Break\;
      }
      \ElseIf{$m = \langle T, \commit, k \rangle$}{
        $\commitVC[j] \leftarrow k$\;
      }
    }

    \ForAll{$\partj \in \partitions(\tx.\RS \cup \tx.\WS)$}{
      \Send{$\DECIDE(\tx, \commitVC,\outcome)$} \KwTo $\partj$\;
    }

    \Return{$\outcome$}\;
  }

  \smallskip

  % Prepare
  \SubAlgo{\WhenReceived $\PREPARE(T, \localRS, \localWS, \localVdep)$ \KwFrom $c_j$}{
    \If{$
      (\exists T'.\ (\langle T', \pending, \localRS', \localWS' \rangle \in \cqueue$
      \begin{tabularx}{\linewidth}{l}
        \quad\quad\quad$\vee\ \langle T', \ready, \_, \_, \_ \rangle \in \cqueue)$\\
        \quad\quad\quad$\wedge\
        (\localWS' \cap \localRS \ne \emptyset
          \wedge \localRS' \cap \localWS \ne \emptyset)$\\
        $\vee \left(
          \exists x. \ \langle x, vsn \rangle \in \localRS
          \wedge (\VersionLog[x].\last.\Vcomm[i] > vsn)\right)$\\
      \end{tabularx}\label{alg:ser_conflict_check}
    }{
      \Send{$\VOTE(t, \abort)$} \KwTo $c_j$\;\label{alg:ser_vote_abort}
      \Return\;
    }

    $\lastprep \leftarrow \lastprep + 1$\;
    $\cqput(\tx, \pending, \localRS, \localWS)$\;
    \Send{$\VOTE(\tx, \commit, \lastprep)$} \KwTo $c_j$\;
  }

  \smallskip

  % Decide
  \SubAlgo{\WhenReceived $\DECIDE(T, \commitVC, \outcome)$ \KwFrom $c_j$}{
    \uIf{$\outcome = \commit$}{
      $\cqupdate(\langle \tx, \ready, \_, \_, \commitVC\rangle)$\;
    }
    \Else{
      $\cqremove(\tx)$\;
    }
  }

  \smallskip

  % Queue head
  \SubAlgo{\Upon $\langle T, \ready, \_, \localWS, \commitVC\rangle = \cqhead()$}{
    \ForAll{$\{\langle x , v \rangle \mid \langle x , v \rangle \in \localWS \wedge \partitionof(x) = i\}$} {
        $\vlapply(\langle x , v , \commitVC \rangle)$\;
    }

    $\mrvc \leftarrow \max(\mrvc,\commitVC)$\;
    $\cladd(T, \mrvc)$\;
    $\cqremove(T)$\;
  }
\end{algorithm}
\caption{Serialisable termination protocol at server $s_i$}
\label{fig:ser_termination}
\end{figure}

\clearpage

\section{Read Committed}
\label{appendix:rc}

Read Committed (RC) is the weakest consistency model that satisfies the \emph{isolation} property required by ACID transactions. It forbids concurrent transactions from observing any data that has not been committed, but it does not place any restriction on the ordering of transactions, and does not preclude write-write conflicts. Thus, transactions may be ordered in any way. Figure~\ref{fig:rc-prot-ds-table} shows a summary of the data structures involved in the protocol.

\begin{figure}[h]
\noindent\adjustbox{max width=\paperwidth}{\footnotesize
\begin{tabularx}{\linewidth}{|c|p{5.5cm}|X|}
  \hline
  \multicolumn{3}{|c|}{\textbf{Variables at a server $s_i$}} \\
  \hline
  \textbf{Name} & \textbf{Domain} & \textbf{Description} \\
  \hline
  $\CommitQueue$

  & Sequence$[\langle \transtype, {\sf State}, {\sf WriteSet} \rangle]$ where ${\sf State}=\{\pending,\ready\}$

  & Queue containing information about update transactions trying to commit at the server. \\
  \hline
  Database

  & Set$[\langle {\sf Object} ,{\sf Value}\rangle]$

  & Set representing the key-value store as a mapping from objects to values. \\
  \hline\hline
  \multicolumn{3}{|c|}{\textbf{Context for a transaction $T$ at a client $c_i$}} \\
  \hline
  $T.\WS$ & ${\sf WriteSet}$ & Write-set of $T$. \\
  \hline
\end{tabularx}
}
\caption{List of variables used in the Read Committed protocol, where ${\sf WriteSet} = {\sf Set}[\langle \keytype, \valuetype \rangle]$.}
\label{fig:rc-prot-ds-table}
\end{figure}

Since transactions only need to observe the last committed version of an object, it is sufficient to store only one version. Thus, we can substitute the $\VersionLog$ mapping with a \emph{Database}, that simply maps an object to its latest version. In addition, transactions don't need to observe a consistent snapshot of the state of a partition, and therefore we can remove all data structures related to computing a snapshot. This is reflected in the execution of a transaction, as can be seen in Figure~\ref{fig:rc_tx_exection}. A server $s_i$ executing a remote read on behalf of a transaction $\tx$ simply fetches the currently available value of the requested object, and returns it to the client (line~\ref{alg:rc_db_get}).

A protocol satisfying Read Committed still needs to offer atomic visibility. To do so, our implementation uses two-phase commit, guaranteeing that a transaction commits at every partition (line~\ref{alg:rc_commit_start}). Servers that participate during the commit phase always vote $\commit$ (line~\ref{alg:rc_send_vote}), since RC does not preclude write-write conflicts. After a successful commit phase, all servers incorporate the transaction updates to its partition state (line~\ref{alg:rc_kv_apply}).

\begin{figure}[h]
\begin{algorithm}[H]
  \setcounter{AlgoLine}{0}
  %  Start
  \SubAlgo{\Fun ${\tt start}()$}{
    \Return{$\KwSty{new}\ \transtype(\WS= \emptyset)$};
  }

  \smallskip

  % Write
  \SubAlgo{\Fun ${\tt write}(T, x, v)$}{
    $\tx.\WS \leftarrow \left(\tx.\WS\ \backslash\ \{\langle x, \_ \rangle\}\right) \cup \{\langle x, v \rangle\}$\;
  }
\end{algorithm}
\caption{Initialisation of a transaction and update of an object \emph{x} at client $c_i$ under Read Committed.}
\end{figure}

\begin{figure}[t]
\begin{algorithm}[H]

  % Read
  \SubAlgo{\Fun ${\tt read}(T, x)$}{
    \If{$\langle x, v \rangle \in \tx.\WS$}{
      \Return{$v$}\;
    }

    $j \leftarrow \partitionof(x)$\;
    \Send{$\READREQUEST(x)$} \KwTo $s_j$\;
    \Receive{$\READRETURN(v)$} \KwFrom $s_j$\;
    \Return{$v$}\;
  }

  \smallskip

  % ReadRequest
  \SubAlgo{\WhenReceived $\READREQUEST(x)$ \KwFrom $c_j$}{
    \Send{$\READRETURN(\kvget(x))$} \KwTo $c_j$\;\label{alg:rc_db_get}
  }

  \smallskip

    % Commit
  \SubAlgo{\Fun ${\tt commit}(T)$\label{alg:rc_commit_start}}{
    \If{$\ws = \emptyset$}{
      \Return{$\commit$}\;
    }

    \ForAll{$\partj \in \partitions(\tx.\WS)$}{
      \Send{$\PREPARE(T)$} \KwTo $\partj$\;
    }

    $\outcome \leftarrow \commit$\;
    \ForAll{$\partj \in \partitions(\tx.\WS)$}{
      \Receive{$\VOTE(m)$} \KwFrom $\partj$\;
      \If{$m = \langle T, \abort\rangle$}{
        $\outcome \leftarrow \abort$\;
        \Break\;
      }
    }

    \ForAll{$\partj \in \partitions(\tx.\WS)$}{
      \Send{$\DECIDE(\tx, \outcome)$} \KwTo $\partj$\;
    }

    \Return{$\outcome$}\;
  }

    % Prepare
  \SubAlgo{\WhenReceived $\PREPARE(T)$ \KwFrom $c_j$}{
    $\cqput(T, \pending, \WS)$\;
    \Send{$\VOTE(T, \commit)$} \KwTo $c_j$\;\label{alg:rc_send_vote}
  }

  % Decide
  \SubAlgo{\WhenReceived $\DECIDE(T, \outcome)$ \KwFrom $c_j$}{
    \uIf{$\outcome = \commit$}{
      $\cqupdate(\langle T, \ready, \_ \rangle)$\;
    }
    \Else{
      $\cqremove(T)$\;
    }
  }

  \smallskip

  % Queue head
  \SubAlgo{\Upon $\langle T, \ready, \localWS\rangle = \cqhead()$}{
    \ForAll{$\{\langle x, v\rangle \mid \langle x, v \rangle \in \localWS \wedge \partitionof(x) = i\}$}{
      $\kvapply(x, v)$\;\label{alg:rc_kv_apply}
    }
  $\cqremove(T)$\;
  }
\end{algorithm}
\caption{Read Committed execution protocol at server $s_i$}
\label{fig:rc_tx_exection}
\end{figure}
