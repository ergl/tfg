\cleardoublepage
\chapter{Comparison Implementations Pseudocode}
\label{appendix:code}

\todo{Revise after finishing preliminaries chapter}

\section{Serialisability}
\label{appendix:ser}

\todo{Check how much of this we already mentioned in chapter 2.}

\todo{SER requires a total order, but is that among conflicting txs, or even in non-conflicting? => there needs to be a serial execution of committed txs, so transactions that will abort can observe bad data, as long as they don't commit. Does our implementation satisify this?}

\todo{Re-read Making PSI Serialisable}

To implement a protocol that satisfies Serialisability, we expand the protocol explained in Chapter~\ref{protocol_chapter} with the following changes:

\begin{itemize}
  \item When a client receives a $\READRETURN$ message after issuing a read of an object $x$ on behalf of a transaction $\tx$, it incorporates the commit time of $x$ in $\tx$'s \emph{read-set}, $\tx.\RS$ (line~\ref{alg:ser_rs_update}). It suffices to store the $j$-th entry of $x$'s \emph{commit vector}, that is, the \emph{sequence number} assigned to the transaction that wrote the version of $x$ read by $\tx$. The read-set of a transaction is initialised to the empty set (line~\ref{alg:ser_start_tx}) and is kept on the transaction context.
  \item Read-only transactions are also certified using two-phase commit.
  \item The decision wether to commit or abort a given transaction $\tx$ ....
\end{itemize}

\begin{figure}[htb!]
\noindent\adjustbox{max width=\paperwidth}{\footnotesize
\begin{tabularx}{\linewidth}{|c|p{6.5cm}|X|}
  \hline
  \multicolumn{3}{|c|}{\textbf{Variables at a server $s_i$}} \\
  \hline
  \textbf{Name} & \textbf{Domain} & \textbf{Description} \\
  \hline
  $\lastprep$ & {\sf Integer} & The number of update transactions that tried to
  commit at the server.\\
  \hline
  $\CommitLog$
  & ${\sf Sequence}[\langle\transtype,$ $\vctype\ V_c, \vctype\ V_a \rangle]$
  & Log of update transactions committed at the server, ordered by
  $V_c[i]$. Here $V_c$ is the commit vector of $T$ and $V_a$ is the aggregate
  vector of $T$: the join of the commit vectors of all transactions up to $T$ in
  $\CommitLog$.\\
  \hline
  $\LocalTime$ & $\vctype$ & The join of the commit vectors of all
  transactions in $\CommitLog$.\\
  \hline
  $\VersionLog$ & ${\sf Map}[\keytype,$ ${\sf
    Set}[\langle \valuetype\ \val, \vctype\ V_c \rangle]]$ & Database:
  a mapping from objects to lists of pairs of a value and the
  commit vector of the transaction that wrote it. The lists are ordered
  by the $i$-th component of the commit vectors.\\
  \hline
  $\CommitQueue$

  & Sequence$[\langle \transtype, {\sf State}, {\sf ReadSet}, {\sf WriteSet} \rangle]$ where ${\sf State}=\{\pending,\ready\}$

  & Queue containing information about update transactions trying to commit at the server. \\
  \hline\hline
  \multicolumn{3}{|c|}{\textbf{Context for a transaction $T$ at a client $c_i$}} \\
  \hline
  $T.\RS$ & ${\sf ReadSet}$ & Read-set of $T$. \\
  \hline
  $T.\WS$ & ${\sf WriteSet}$ & Write-set of $T$. \\
  \hline
  $T.\hasRead$ & ${\sf Vector}[{\sf Bool}]$ & Mapping showing whether $T$  has
  read a given partition.
\\
  \hline
  $T.\VCaggr$ & $\vctype$ & Snapshot vector: determines snapshots fixed at
  partitions $T$ has read from and possible causal dependencies at all other
  partitions.
  \\
  \hline
  $T.\VCdep$ & $\vctype$ & Dependency vector, representing all causal
  dependencies developed by $T$ during its execution.
\\
  \hline
\end{tabularx}
}
\caption{List of variables used in the Serialisable protocol, where ${\sf ReadSet} = {\sf Set}[\langle \keytype, {\sf Integer}\rangle]$ and ${\sf WriteSet} = {\sf Set}[\langle \keytype, \valuetype \rangle]$}
\label{fig:ser-prot-ds-table}
\end{figure}

\begin{figure}[h]
\begin{algorithm}[H]
  \setcounter{AlgoLine}{0}
  % Start
  \SubAlgo{\Fun ${\tt start}()$}{
    \Return{$\KwSty{new}\ \transtype(
      \WS= \emptyset,
      \RS= \emptyset,
      \hasRead = \vec{\bot},
      \VCaggr = \vec{0},
      \VCdep = \vec{0})$
    };\label{alg:ser_start_tx}
  }

  \smallskip

  % Write
  \SubAlgo{\Fun ${\tt write}(T, x, v)$}{
    $\tx.\WS \leftarrow \left(\tx.\WS\ \backslash\ \{\langle x, \_ \rangle\}\right) \cup \{\langle x,v\rangle\}$\;
  }
\end{algorithm}
\caption{Initialisation of a transaction and update of an object \emph{x} at client $c_i$ under Serialisability.}
\end{figure}

\begin{figure}[h]
\begin{algorithm}[H]
  % Read
  \SubAlgo{\Fun ${\tt read}(T, x)$}{
    \If{$\langle x, v \rangle \in \tx.\WS$}{
      \Return{$v$}\;
    }

    $j \leftarrow \partitionof(x)$\;
    \Send{$\READREQUEST(x, T.\VCaggr, T.\hasRead)$} \KwTo $s_j$\;
    \Receive{$\READRETURN(m)$} \KwFrom $s_j$\;
    \uIf{$m = \abort$} {
      \Throw{$\abort$}\;
    }
    \ElseIf{$m = \langle v,\localVdep,\localVaggr \rangle$}{
      $\tx.\hasRead[j] \leftarrow \true$\;
      $\tx.\RS \leftarrow
        \left(\tx.\RS\ \backslash\ \{\langle x, \_ \rangle\}\right)
        \cup
        \{\langle x, \localVdep[j] \rangle\}$\;\label{alg:ser_rs_update}
      $\tx.\VCdep \leftarrow \max(\tx.\VCdep,\localVdep)$\;
      $\tx.\VCaggr \leftarrow \max(\tx.\VCaggr,\localVaggr)$\;
      \Return{$v$}\;
    }
  }

  \smallskip

  % ReadRequest
  \SubAlgo{\WhenReceived $\READREQUEST(x, \argVCaggr, \argHasRead)$ \KwFrom $c_j$}{
    \uIf{$\argHasRead[i]$} {
      $V \leftarrow \argVCaggr$\;
    }
    \Else{
      \Until{$\mrvc[i] \ge \argVCaggr[i]$}\;
      $r \leftarrow \max\{r \in \CommitLog \mid \forall j.\, \argHasRead[j] {\implies} \left(r.\Vaggr[j] \le \argVCaggr[j]\right)\}$\;
      \If{$r.\Vaggr[i] < \argVCaggr[i]$}{
        \Send{$\READRETURN(\abort)$} \KwTo $c_j$\;
        \Return\;
      }
      $V \leftarrow r.\Vaggr$\;
    }
    $\ver = \max\{\ver \in \VersionLog \mid ver.\Vcomm[i] \le V[i]\}$\;
    \Send{$\READRETURN(\ver.\val, \ver.\Vcomm,V)$} \KwTo $c_j$\;
  }
\end{algorithm}
\caption{Serialisable local and remote read of object \emph{x}}
\end{figure}

\begin{figure}[h]
\begin{algorithm}[H]
  % Commit
  \SubAlgo{\Fun ${\tt commit}(T)$\label{alg:commit_start}}{
    \ForAll{$\partj \in \partitions(\tx.\RS \cup \tx.\WS)$}{
      \Send{$\PREPARE(T, T.\RS, T.\WS, \VCdep)$} \KwTo $\partj$\;
    }

    $\commitVC \leftarrow \tx.\VCdep$\;
    $\outcome \leftarrow \commit$\;

    \ForAll{$\partj \in \partitions(\tx.\RS \cup \tx.\WS)$}{
      \Receive{$\VOTE(m)$} \KwFrom $\partj$\;
      \uIf{$m = \langle T, \abort \rangle$}{
        $\outcome \leftarrow \abort$\;
        \Break\;
      }
      \ElseIf{$m = \langle T, \commit, k \rangle$}{
        $\commitVC[j] \leftarrow k$\;
      }
    }

    \ForAll{$\partj \in \partitions(\tx.\RS \cup \tx.\WS)$}{
      \Send{$\DECIDE(\tx, \commitVC,\outcome)$} \KwTo $\partj$\;
    }

    \Return{$\outcome$}\;
  }

  \smallskip

  % Prepare
  \SubAlgo{\WhenReceived $\PREPARE(T, \localRS, \localWS, \localVdep)$ \KwFrom $c_j$}{
    \If{$
      (\exists T'.\ (\langle T', \pending, \localRS', \localWS' \rangle \in \cqueue$
      \begin{tabularx}{\linewidth}{l}
        \quad\quad\quad$\vee\ \langle T', \ready, \_, \_, \_ \rangle \in \cqueue)$\\
        \quad\quad\quad$\wedge\
        (\localWS' \cap \localRS \ne \emptyset
          \wedge \localRS' \cap \localWS \ne \emptyset)$\\
        $\vee \left(
          \exists x. \ \langle x, vsn \rangle \in \localRS
          \wedge (\VersionLog[x].\last.\Vcomm[i] > vsn)\right)$\\
      \end{tabularx}
    }{
      \Send{$\VOTE(t, \abort)$} \KwTo $c_j$\;
      \Return\;
    }

    $\lastprep \leftarrow \lastprep + 1$\;
    $\cqput(\tx, \pending, \localRS, \localWS)$\;
    \Send{$\VOTE(\tx, \commit, \lastprep)$} \KwTo $c_j$\;
  }

  \smallskip

  % Decide
  \SubAlgo{\WhenReceived $\DECIDE(T, \commitVC, \outcome)$ \KwFrom $c_j$}{
    \uIf{$\outcome = \commit$}{
      $\cqupdate(\langle \tx, \ready, \_, \_, \commitVC\rangle)$\;
    }
    \Else{
      $\cqremove(\tx)$\;
    }
  }

  \smallskip

  % Queue head
  \SubAlgo{\Upon $\langle T, \ready, \_, \localWS, \commitVC\rangle = \cqhead()$}{
    \ForAll{$\{\langle x , v \rangle \mid \langle x , v \rangle \in \localWS \wedge \partitionof(x) = i\}$} {
        $\vlapply(\langle x , v , \commitVC \rangle)$\;
    }

    $\mrvc \leftarrow \max(\mrvc,\commitVC)$\;
    $\cladd(T, \mrvc)$\;
    $\cqremove(T)$\;
  }
\end{algorithm}
\caption{Serialisable termination protocol at server $s_i$}
\end{figure}

\clearpage

\section{Read Committed}
\label{appendix:rc}

Read Committed (RC) is the weakest consistency criterion that satisfies the \emph{isolation} property required by ACID transactions. It forbids concurrent transactions from observing any data that has not been committed. It doesn't place any restriction on the ordering of transactions, and doesn't preclude write-write conflicts. Thus, transactions may be ordered in any way. Figure~\ref{fig:rc-prot-ds-table} shows a summary of the data structures involved in the protocol.

\begin{figure}[h]
\noindent\adjustbox{max width=\paperwidth}{\footnotesize
\begin{tabularx}{\linewidth}{|c|p{5.5cm}|X|}
  \hline
  \multicolumn{3}{|c|}{\textbf{Variables at a server $s_i$}} \\
  \hline
  \textbf{Name} & \textbf{Domain} & \textbf{Description} \\
  \hline
  $\CommitQueue$

  & Sequence$[\langle \transtype, {\sf State}, {\sf WriteSet} \rangle]$ where ${\sf State}=\{\pending,\ready\}$

  & Queue containing information about update transactions trying to commit at the server. \\
  \hline
  Database

  & Set$[\langle {\sf Object} ,{\sf Value}\rangle]$

  & Set representing the key-value store as a mapping from objects to values. \\
  \hline\hline
  \multicolumn{3}{|c|}{\textbf{Context for a transaction $T$ at a client $c_i$}} \\
  \hline
  $T.\WS$ & ${\sf WriteSet}$ & Write-set of $T$. \\
  \hline
\end{tabularx}
}
\caption{List of variables used in the Read Committed protocol, where ${\sf WriteSet} = {\sf Set}[\langle \keytype, \valuetype \rangle]$.}
\label{fig:rc-prot-ds-table}
\end{figure}

Since transactions only need to observe the last committed version of an object, it is sufficient to store only one version per key. In addition, transactions don't need to observe a consistent snapshot of the state of a partition, and therefore we can remove all data structures related to computing a snapshot. This is reflected in the execution of a transaction, as can be seen in Figure~\ref{fig:rc_tx_exection}. A server $s_i$ executing a remote read on behalf of a transaction $\tx$ simply fetches the currently available value of the requested object, and returns it to the client (line~\ref{alg:rc_db_get}).

A protocol satisfying Read Committed still needs to offer atomic visibility. To do so, our implementation uses two-phase commit, guaranteeing that a transaction commits at every partition (line~\ref{alg:rc_commit_start}). Servers that participate during the commit phase always vote $\commit$ (line~\ref{alg:rc_send_vote}), since RC doesn't preclude write-write conflicts. After a successful commit phase, all partitions incorporate the transaction updates to its storage state (line~\ref{alg:rc_kv_apply}).

\begin{figure}[h]
\begin{algorithm}[H]
  \setcounter{AlgoLine}{0}
  %  Start
  \SubAlgo{\Fun ${\tt start}()$}{
    \Return{$\KwSty{new}\ \transtype(\WS= \emptyset)$};
  }

  \smallskip

  % Write
  \SubAlgo{\Fun ${\tt write}(T, x, v)$}{
    $\tx.\WS \leftarrow \left(\tx.\WS\ \backslash\ \{\langle x, \_ \rangle\}\right) \cup \{\langle x, v \rangle\}$\;
  }
\end{algorithm}
\caption{Initialisation of a transaction and update of an object \emph{x} at client $c_i$ under Read Committed.}
\end{figure}

\begin{figure}[t]
\begin{algorithm}[H]

  % Read
  \SubAlgo{\Fun ${\tt read}(T, x)$}{
    \If{$\langle x, v \rangle \in \tx.\WS$}{
      \Return{$v$}\;
    }

    $j \leftarrow \partitionof(x)$\;
    \Send{$\READREQUEST(x)$} \KwTo $s_j$\;
    \Receive{$\READRETURN(v)$} \KwFrom $s_j$\;
    \Return{$v$}\;
  }

  \smallskip

  % ReadRequest
  \SubAlgo{\WhenReceived $\READREQUEST(x)$ \KwFrom $c_j$}{
    \Send{$\READRETURN(\kvget(x))$} \KwTo $c_j$\;\label{alg:rc_db_get}
  }

  \smallskip

    % Commit
  \SubAlgo{\Fun ${\tt commit}(T)$\label{alg:rc_commit_start}}{
    \If{$\ws = \emptyset$}{
      \Return{$\commit$}\;
    }

    \ForAll{$\partj \in \partitions(\tx.\WS)$}{
      \Send{$\PREPARE(T)$} \KwTo $\partj$\;
    }

    $\outcome \leftarrow \commit$\;
    \ForAll{$\partj \in \partitions(\tx.\WS)$}{
      \Receive{$\VOTE(m)$} \KwFrom $\partj$\;
      \If{$m = \langle T, \abort\rangle$}{
        $\outcome \leftarrow \abort$\;
        \Break\;
      }
    }

    \ForAll{$\partj \in \partitions(\tx.\WS)$}{
      \Send{$\DECIDE(\tx, \outcome)$} \KwTo $\partj$\;
    }

    \Return{$\outcome$}\;
  }

    % Prepare
  \SubAlgo{\WhenReceived $\PREPARE(T)$ \KwFrom $c_j$}{
    $\cqput(T, \pending, \WS)$\;
    \Send{$\VOTE(T, \commit)$} \KwTo $c_j$\;\label{alg:rc_send_vote}
  }

  % Decide
  \SubAlgo{\WhenReceived $\DECIDE(T, \outcome)$ \KwFrom $c_j$}{
    \uIf{$\outcome = \commit$}{
      $\cqupdate(\langle T, \ready, \_ \rangle)$\;
    }
    \Else{
      $\cqremove(T)$\;
    }
  }

  \smallskip

  % Queue head
  \SubAlgo{\Upon $\langle T, \ready, \localWS\rangle = \cqhead()$}{
    \ForAll{$\{\langle x, v\rangle \mid \langle x, v \rangle \in \localWS \wedge \partitionof(x) = i\}$}{
      $\kvapply(x, v)$\;\label{alg:rc_kv_apply}
    }
  $\cqremove(T)$\;
  }
\end{algorithm}
\caption{Read Committed execution protocol at server $s_i$}
\label{fig:rc_tx_exection}
\end{figure}
